\documentclass[a4paper]{report}


\def\authorname{Joseph Ramsay}
\def\departmentname{LINZ Data Service}
\def\companyname{Land Information New Zealand}


\title{Incremental Testing}

\begin{document}

\maketitle

\section*{Part 1. Introduction}

This document describes the python scripts used to facilitate testing for the LDS incremental feature. these scripts can be used as the basis for further development or simply as an instruction into one method to interact with LDS Incremental.

\section{Part 2. Feature Description}

\subsection*{LDS Incremental Service}
Whenever an LDS vector layer is updated with new data the existing data is not deleted. Instead the change-set or difference between the existing data and the new data is determined and stored in as versioned information. This versioned information is then able to provide layer data at any version (users can’t do this currently) as well as the change-set between any two layer versions. The change-set is a group of feature modifications for a given layer from one user defined revision to another revision. If this range from start to end includes multiple revisions, the change-set is only provided from the first version to the last version - intermediate changed are not provided. 

Within LDS the only layers that can be versioned are the ones that have an integer primary key feature identifier.  These identifiers are used for doing feature by feature comparisons. Currently only the LDS Property and Ownership layers have these identifiers. 

Note: In the case of the Property and Ownership layers (which are derived from the Landonline system) the primary key feature identifiers are not classed as Static Unique Feature IDs (SUFID). This means that these primary key identifiers in some cases are re-used by new feature objects. This can sometimes cause issues for users who are trying to apply change-set update logic to external systems. In some cases Landonline feature are reported as updated, when in fact a new real world object has been created. LINZ is currently investigating options to guarantee SUFID from Landonline data.

\section*{Part 3. Test Script}
Three main processes are abstracted in the test script; the WFS datastore object representing the requested page, a database object and a configuration object representing user defined settings.

\section*{Part 4. Test Suite} 

\end{document}
